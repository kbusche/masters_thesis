\chapter{Frequency Modulated Continuous Wave Radar}
As the name Frequency Modulated Continuous Wave (FMCW) implies, a FMCW radar is
a continuous time system which transmits and receives a periodic signal whose 
frequency has been modulated. As a periodic signal, the transmitted signal has
the complex form (with unit-normalized amplitude)
\begin{equation}
	\label{eq:complex-sinusoid}
	p(t) = e^{j2\pi f(t)t}.
\end{equation}
The typical frequency modulation used in FMCW radar systems is the sawtooth
modulation, given by
\cite{iovescufundamentals, wang2008digital}
\begin{equation}
	\label{eq:sawtooth}
	f(t) = f_c + \alpha t, \quad 0<t<T_c 
\end{equation}
where $\alpha > 0$ is the chirp-rate  $\frac{df}{dt}$, $f_c$ is the base
carrier frequency (e.g. 77 GHz), and $T_c$ is the period of the chirp. The
maximum frequency of each chirp is thus
\begin{equation}
	f_{max} \triangleq f_c + \alpha T_c,
\end{equation}
and the bandwidth $B$ of the signal is
\begin{equation}
	B = f_{max} - f_c = \alpha T_c.
\end{equation}

Combining the sawtooth frequency modulation (\ref{eq:sawtooth}) with the complex
sinusoid (\ref{eq:complex-sinusoid}), we get
the transmitted pulse as
\begin{equation}
	p(t) = e^{j(2\pi f_c t+ \pi \alpha t^2)}.
\end{equation}
