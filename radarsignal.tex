\chapter{Frequency Modulated Continuous Wave Radar}
As the name Frequency Modulated Continuous Wave (FMCW) implies, a FMCW radar is
a continuous time system which transmits and receives a periodic signal whose 
frequency has been modulated. As a periodic signal, the transmitted signal has
the complex form (with unit-normalized amplitude)
\begin{equation}
	\label{eq:complex-sinusoid}
	p(t) = e^{j2\pi f(t)t}.
\end{equation}
The typical frequency modulation used in FMCW radar systems is the sawtooth
modulation, given by
\cite{iovescufundamentals, wang2008digital}
\begin{equation}
	\label{eq:sawtooth}
	f(t) = f_c + \alpha t, \quad 0<t<T_c 
\end{equation}
where $\alpha > 0$ is the chirp-rate  $\frac{df}{dt}$, $f_c$ is the base
carrier frequency (e.g. 77 GHz), and $T_c$ is the period of the chirp. The
maximum frequency of each chirp is thus
\begin{equation}
	f_{max} \triangleq f_c + \alpha T_c,
\end{equation}
and the bandwidth $B$ of the signal is
\begin{equation}
	B = f_{max} - f_c = \alpha T_c.
\end{equation}

Combining the sawtooth frequency modulation (\ref{eq:sawtooth}) with the complex
sinusoid (\ref{eq:complex-sinusoid}), we get
the transmitted (TX) signal as
\begin{equation}
	p(t) = e^{j(2\pi f_c t+ \pi \alpha t^2)}.
\end{equation}

\section{Range Measurement}
Consider a target at a distance $d$ from the radar, such that the transmitted
(RX) signal reflects off the target and returns to the radar. This received signal
will be a time delayed version of the TX signal, where the time delay $\tau$ is
given by
\begin{equation}
	\tau = \frac{2d}{c}
\end{equation}
where $c$ is the speed of light. The RX signal thus has the form
\begin{equation}
	p(t-\tau)=e^{j(2\pi f_c (t-\tau) + \pi\alpha (t-\tau)^2)}.
\end{equation}
To recover $\tau$, and subsequently $d$, we define a new dechirped signal $r(t)$
as the product of the transmitted signal with the complex conjugate of the
received signal
\begin{align}
	r(t) &\triangleq p(t)p^*(t-\tau) \\
	&= e^{j(2\pi f_c t + \pi \alpha t^2)}e^{-j[2\pi f_c (t-\tau) + \pi\alpha (t-\tau)^2 ]} \\
	&= e^{j(2\pi f_c \tau - \pi \alpha \tau^2)}e^{j2\pi\alpha\tau t}.\label{eq:range}
\end{align}
Note, the first exponential in (\ref{eq:range}) only depends on $\tau$, so it is
a constant phase term. However, the second term varies according to a constant
frequency (named the beat frequency) $f_b$
\begin{equation}
	f_b \triangleq \alpha \tau.
\end{equation}
The maximum beat frequency occurs when $\tau = T_c$, as any $\tau > T_c$ will
appear as $\tau^*$
\begin{equation}
	\tau^* = \tau - T_c, \quad \tau > T_c
\end{equation}
and thus the recovered distance $d^*$ will be less than the true range the
target is from the radar. From this, we can get our maximum recoverable distance
for a given chirp period
\begin{equation}
	d_{max} = \frac{c T_c}{2}.
\end{equation}
To recover the beat frequency $f_b$ from the dechirped signal, $r(t)$, we can
simply use the Fourier Transform to get
\begin{align}
	R(f) &= \int_{-\infty}^{\infty} r(t) e^{-j2 \pi ft} dt\\
	&= \int_{-\infty}^{\infty} e^{j(2\pi f_c \tau - \pi \alpha \tau^2)}e^{j2\pi\alpha\tau t} e^{-j2\pi ft}dt\\
	&= e^{j(2\pi f_c \tau - \pi \alpha
	\tau^2)}\int_{-\infty}^{\infty}e^{-j2\pi(f - \alpha\tau ) t} dt\\
	&= e^{j(2\pi f_c \tau - \pi \alpha \tau^2)}\delta (f - \alpha \tau).
\end{align}

