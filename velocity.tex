\chapter{Estimating Velocities}
To determine the velocity of an object, we can compare the phases of two chirps
reflected from the moving object. Recall from (\ref{eq:range}) that the first
exponential term in $r(t)$ has no dependence on $t$, so it fully describes the phase. We
can approximate this phase as a linear function of the time delay $\tau$, 
\begin{align}
	\phi(r(t)) &= 2\pi f_c \tau - \pi \alpha \tau^2 + 2\pi\alpha\tau t \\
	\implies \angle r(t) &\approx 2\pi f_c \tau.
\end{align}
For an object moving with velocity $v$ and initial range $d_0$, the time delay
$\tau$ is given by 
\begin{equation}
	\tau = \frac{2(d_0 + vt)}{c}, \label{eq:moving-tau}
\end{equation}
so the linear phase approximation becomes
\begin{equation}
	\phi(r(t)) \approx \frac{4\pi(d_0 + vt)}{\lambda},
\end{equation}
where $\lambda = \frac{c}{f_c}$ the wavelength of
the carrier signal. Now, the phase difference $\Delta \phi$ between two
dechirped signals separated by
$T_c$ is
\begin{equation}
	\Delta\phi = \frac{4\pi \Delta d}{\lambda},
\end{equation}
where $\Delta d$ is the distance the object travels over $T_c$ between the two
chirps. For an object traveling at a velocity $v$, this distance is
\begin{equation}
	\Delta d = v T_c, 
\end{equation}
so the phase difference becomes
\begin{equation}
	\label{phase_diff}
	\Delta\phi = \frac{4\pi v T_c}{\lambda}.
\end{equation}
However, phase difference is inherently ambiguous unless $|\Delta\phi|<\pi$, so
we find the maximum velocity measurable by two chirps spaced $T_c$ apart is
given by
\begin{equation}
	v_{max} = \frac{\lambda}{4 T_c}.
\end{equation} 

This two-chirp approach fails if we have multiple objects moving with different
velocities, but the same range at the time of measurement. The reflected chirps
will produce idential beat frequencies in this scenario, so the range processing
Fourier Transform will result in a single peak representing the combined signals
from the equidistant objects. In this case, we can use $K$ equally spaced chirps 
transmitted by the radar over a chirp frame $T_f = KT_c$. 

\begin{figure}[h]
	\centering
	\begin{tikzpicture}
		\begin{axis}[
		axis x line = bottom,
		axis y line = left,
		xlabel = $t$,
		ylabel=$f(t)$,
		xmin=0, xmax=8.5,
		ymin=0.5, ymax=5.2,
		xtick=\empty,
		ytick=\empty,
		xlabel near ticks,
		ylabel near ticks,
		extra x ticks={2,4,8},
		extra x tick labels={$T_c$, $2T_c$, $T_f =
			KT_c$},
		extra y ticks={0.5, 1, 5},
		extra y tick labels={$0$, $f_c$, $f_{max}$},
		]
		\addplot [
		domain=0:8,
		samples=100,
		color=black,
		]
		{(1 + 2*x)*(x<2) + (1+2*(x-2))*(x>2)*(x<4) + (1+2*(x-4))*(x>4)*(x<6) + (1+2*(x-6))*(x>6)*(x<8)};
		\end{axis}
	\end{tikzpicture}
	\label{fig:chirp-frame}
	\caption{Chirp frame $T_f$ consisting of $K$ chirps}
\end{figure}

Now we will need to use the full sawtooth frequency modulation as described in
(\ref{eq:sawtooth}), so $p(t)$ is
\begin{equation}
	p(t) = e^{j(2\pi f_c t+ \pi \alpha t^2 - 2\pi\alpha kT_c t)}.
\end{equation}
Then, the dechirped signal $r(t)$ will be
\begin{equation}
	r(t) = e^{j(2\pi f_c \tau - \pi \alpha \tau^2 + 2\pi\alpha\tau (t-kT_c))}.
\end{equation}
Again, we approximate by eliminating the quadratic components to get
\begin{equation}
	\phi (r(t)) \approx 2\pi (f_c + \alpha t_k)\tau,
	\label{eq:phase-approx}
\end{equation}
where 
\begin{equation}
	t_k \triangleq t - kT_c \quad \text{for }k\in\mathbb{Z}.
\end{equation}
Substituting (\ref{eq:moving-tau}) into (\ref{eq:phase-approx}) gives
\begin{align}
	\phi (r(t)) &\approx \frac{4\pi}{c}(f_c d_0 + f_c vt + \alpha d_0 t_k)\\
	&= 2\pi (f_c \tau_0 + f_d t + t_k f_{\tau})\\
	&= 2\pi (f_c \tau_0 + f_d k T_c + (f_{\tau} + f_d) t_k)
\end{align}
where $\tau_0$ is the initial time delay, $f_d$ is the Doppler Frequency, and
$f_{\tau}$ is the range-beat-frequency, as defined below.
\begin{align}
	\tau_0 &\triangleq \frac{2d_0}{c}\\
	f_d &\triangleq \frac{2v}{c} f_c\\
	f_{\tau} &\triangleq \alpha\tau_0
\end{align}
For each $k$-th chirp, we compute $R_r(f,k)$ as in (\ref{eq:range-ft}), which
contains the range information for the objects,
\begin{align}
	R_r(f,k) &= \int_{\tau}^{T_c} r(t_k)e^{-j2\pi f t_k}dt_k\\
	&= \int_{\tau}^{T_c} e^{j\phi(r(t_k))}e^{-j2\pi f t_k}dt_k\\
	&= \int_{\tau}^{T_c} e^{j2\pi (f_c \tau_0 + f_d k T_c + (f_{\tau} + f_d) t_k)}e^{-j2\pi f t_k}dt_k.
\end{align}
The absolute value $|R_r(f,k)|$ is obtained for $f = f_d + f_\tau$, and, in
general, $f_\tau >> f_d$, so we can resolve the ranges as before.
Now, since the term $f_dkT_c$ depends on $k$, we see $R_r(f,k)$ is a 
function of $k$, with sampling period $T_c$, observed $K$ consecutive times over the course of the chirp frame
$T_f$. Each of these $k$ spectrums will have a different phase which combines the
phase contributions from each object at the corresponding range with different
velocities. This spectrum is described by the following Discrete Fourier Transform
\begin{equation}
	\label{eq:velocity-fft}
	R(f, n) = \sum_{k=0}^{K-1}R_r(f, k) e^{-j\frac{2\pi}{K}kn},
\end{equation}
which achieves maximum absolute value (i.e. a peak) when 
\begin{equation}
	n = f_d T_c.
\end{equation}

In the case of multiple objects at the same range, but with different
velocities, we will have peaks corresponding to the Doppler Frequencies of each
object, enabling us to resolve the different velocities. Since we are using the
DFT, we can derive the velocity resolution as we did the range resolution in
(\ref{eq:range-res}), where the observation window here is the frame time $T_f$,
\begin{equation}
	\Delta v = \frac{c}{2 K T_c f_c} = \frac{c}{2 f_c T_f}.
\end{equation}

This spectrum is computed using the Fast Fourier Transform method, so we call
this step the Doppler-FFT. By reordering the samples of $r(t)$ into a 2D matrix
$\bm{R}$ where each row is the samples from one chirp period $T_c$ with $K$ rows
(i.e. $K$ chirps), we can compute the Range-FFT and Doppler-FFT as a 2D FFT over
the data matrix $\bm{R}$. 
